\documentclass{article}
\usepackage{hyperref}
\usepackage{geometry}
\usepackage{tabularray}
\usepackage{xcolor}
\usepackage{fontspec}
\usepackage{soul}
\usepackage{ifthen}
\usepackage{fancyhdr}
\usepackage{roboto}
\usepackage{xstring}
\usepackage{seqsplit}
\usepackage{graphicx}

\geometry{top=0.75in, bottom=0.75in, left=0.25in, right=0.25in}
\geometry{landscape}

% The url formatting to have it set for blue
\hypersetup{
    colorlinks=true,
    linkcolor=blue,
    filecolor=magenta,
    urlcolor=blue,
    pdfborderstyle={/S/U/W 1}
}

\let\oldhref\href
\renewcommand{\href}[2]{\oldhref{#1}{\ul{#2}}}

% Status Image
% This processes the status field and returns the image circle. This is based on the status column of a check
% The images must be included in the ./template folder on the base repo.
% The only argument is the status of the check
\newcommand{\statusimage}[1]{%
    \centering
    \raisebox{-0.3\height}{%
        \IfStrEqCase{#1}{%
                {pass}{\includegraphics[width=.6cm, height=.6cm]{green-circle}}%
                {warn}{\includegraphics[width=.6cm, height=.6cm]{yellow-circle}}%
                {fail}{\includegraphics[width=.6cm, height=.6cm]{red-circle}}%
                {ignore}{\includegraphics[width=.6cm, height=.6cm]{white-circle}}%
                {error}{\includegraphics[width=.6cm, height=.6cm]{error-image}}%
        }[{#1}]%
    }%
}

% Plugin - Pluygin Link
% 1. URL
% 2. Text to display
% If the URL is empty, it will just display the text
\newcommand{\pluginlink}[2]{
    \ifthenelse{\equal{#1}{}}
    {#2} % if the first argument is empty, then just display the text
    {\href{#1}{#2}} % if the first argument is not empty, then display the text as a link
}

% Plugin - Get Plugin Version
% 1. (plugins.isUpdated) A boolean to check if the plugin is updated
% 2. (plugins.installedVersion) The installed version
% 3. (plugins.latestVersion) The latest version
% If the plugin is updated, it will return the installed version
% If the plugin is not updated, it will return the installed version in red
\newcommand{\getPluginVersion}[3]{%
    \ifthenelse{\equal{#1}{true}}% plugins.isUpdated == true
        {#2}% return plugins.installedVersion
        {% else
            \ifthenelse{\equal{#3}{}}% if plugins.latestVersion == ""
            {#2}% return plugins.installedVersion
            {\textcolor{red}{#2}}% else return plugins.installedVersion in red
        }%
}

% Plugin - Get Latest Version
% 1. (plugins.latestVersion) The latest version
% 2. (plugins.latestReleaseDate) The latest release date
\newcommand{\getLatestVersion}[2]{
    #1 % return plugins.latestVersion
    % adds the release date if it exists
    \ifthenelse{\equal{#2}{}}%
        {}%
        {\footnotesize(#2)} % if plugins.latestReleaseDate exists, add it in parentheses
}

% Setting this to a Mattermost approved font and making it the default
\setsansfont{Roboto}
\renewcommand*\familydefault{\sfdefault}

% Header for the document
\pagestyle{fancy}
\fancyhf{}
\lhead{$metadata.companyName$}
\rhead{$metadata.date$}


\begin{document}

    \section*{Mattermost Healthcheck Report}
    This is an auto generated report from the healthcheck tool.

    \section{Environment Checks}\label{sec:environment-checks}
        \begin{tblr}{|Q[l,m]|Q[l,m]|Q[c,m]|Q[l,m,3cm]|Q[l,m,3cm]|X[l,m]|}

            \hline
            ID & SEV & STATUS & NAME & RESULT & DESCRIPTION \\
            \hline

            $for(environment)$
                \MakeUppercase{$environment.id$}  &
                $environment.severity$ &
                \statusimage{$environment.status$} &
                $environment.name$ &
                $environment.result$ &
                $environment.description$ \\

                % adds the horizontal line to the bottom of the column
                \hline
            $endfor$

        \end{tblr}

    \section{Packet Checks}\label{sec:packet-checks}
        \begin{tblr}{|Q[l,m]|Q[l,m]|Q[c,m]|Q[l,m,3cm]|Q[l,m,3cm]|X[l,m]|}

            \hline
            ID & SEV & STATUS & NAME & RESULT & DESCRIPTION \\
            \hline

            $for(packet)$
                \MakeUppercase{$packet.id$}  &
                $packet.severity$ &
                \statusimage{$packet.status$} &
                $packet.name$ &
                $packet.result$ &
                $packet.description$ \\
                % adds the horizontal line to the bottom of the column
                \hline
            $endfor$

        \end{tblr}

    \section{Configuration Checks}\label{sec:configuration-checks}
        \begin{tblr}{|Q[l,m]|Q[l,m]|Q[c,m]|Q[l,m,3cm]|Q[l,m,3cm]|X[l,m]|}

            \hline
            ID & SEV & STATUS & NAME & RESULT & DESCRIPTION \\
            \hline

            $for(config)$
                \MakeUppercase{$config.id$}  &
                $config.severity$ &
                \statusimage{$config.status$} &
                $config.name$ &
                $config.result$ &
                $config.description$ \\
                % adds the horizontal line to the bottom of the column
                \hline
            $endfor$

        \end{tblr}

    \section{Mattermost Log}\label{sec:mattermost-log}

        \subsection{Checks}\label{subsec:checks}
            \begin{tblr}{|Q[l,m]|Q[l,m]|Q[c,m]|Q[l,m,3cm]|Q[l,m,3cm]|X[l,m]|}

                \hline
                ID & SEV & STATUS & NAME & RESULT & DESCRIPTION \\
                \hline

                $for(mattermostLog)$
                    \MakeUppercase{$mattermostLog.id$} &
                    $mattermostLog.severity$ &
                    \statusimage{$mattermostLog.status$} &
                    $mattermostLog.name$ &
                    $mattermostLog.result$ &
                    $mattermostLog.description$ \\
                    \hline
                $endfor$

            \end{tblr}

    \subsection{Top Logs}\label{subsec:top-logs}

        \begin{tblr}{|Q[l,m]|Q[l,m]|X[m]|Q[l,m]|}

            \hline
            COUNT & LEVEL & MESSAGE & CALLER \\
            \hline

            $for(topLogs)$
                $topLogs.count$ &
                $topLogs.level$ &
                $topLogs.msg$ &
                $topLogs.caller$  \\
                \hline
            $endfor$

        \end{tblr}



    \section{Plugins}\label{sec:plugins}

        \subsection{Active}\label{subsec:active}
            \begin{tblr}{|Q[l,m]|Q[l,m]|Q[l,m]|Q[l,m]|Q[l,m]|}
                \hline
                ID & NAME & INSTALLED VERSION & LATEST VERSION & SUPPORT \\
                \hline
                $for(plugins)$
                    $if(plugins.active)$
                        $plugins.pluginID$ &
                        \pluginlink{$plugins.pluginURL$}{$plugins.pluginName$} &
                        \getPluginVersion{$plugins.isUpdated$}{$plugins.installedVersion$}{$plugins.latestVersion$} &
                        \getLatestVersion{$plugins.latestVersion$}{$plugins.latestReleaseDate$} &
                        $plugins.supportLevel$ \\
                        \hline

                    $endif$
                $endfor$

            \end{tblr}

    \subsection{Inactive}\label{subsec:inactive}

    \begin{tblr}{|Q[l,m]|Q[l,m]|Q[l,m]|Q[l,m]|Q[l,m]|}
        \hline
        ID & NAME & INSTALLED VERSION & LATEST VERSION & SUPPORT \\
        \hline

        $for(plugins)$
            % skipping active plugins since those happen above
            $if(plugins.active)$
            $else$
                $plugins.pluginID$ &
                \pluginlink{$plugins.pluginURL$}{$plugins.pluginName$} &
                \getPluginVersion{$plugins.isUpdated$}{$plugins.installedVersion$}{$plugins.latestVersion$} &
                \getLatestVersion{$plugins.latestVersion$}{$plugins.latestReleaseDate$} &
                $plugins.supportLevel$ \\
                \hline
            $endif$
          $endfor$
    \end{tblr}

\end{document}